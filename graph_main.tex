\documentclass[13pt]{beamer}

\usepackage{Handle}

\DeclareFontFamily{U}{BOONDOX-calo}{\skewchar\font=45}
\DeclareFontShape{U}{BOONDOX-calo}{m}{n}{
  <-> s*[1.05] BOONDOX-r-calo}{}
\DeclareMathAlphabet{\mathcalboondox}{U}{BOONDOX-calo}{m}{n}
\newcommand{\bondox}{\mathcalboondox}

\newcommand{\Title}{Two betweenness centrality
measures based on Randomized
Shortest Paths}
\newcommand{\Subtitle}{Prova Integrativa - Complessità nei Sistemi e nelle Reti}
\newcommand{\YourName}{Matteo Bonfadini}
\newcommand{\Institute}{ Politecnico di Milano}
\newcommand{\ImageUrl}{Images/graph.jpg}

\begin{document}
    \begin{frame}[noframenumbering,plain,t]

        \begin{tikzpicture}[remember picture, overlay]
            \node[opacity = 0.8] at (current page.center)
            {
                \includegraphics[height = 0.75\paperwidth, width = 1\paperwidth]{Images/bg9.png}
            };
        \end{tikzpicture}

        \MakeTitle

    \end{frame}


    \begin{frame}{Outline}
        \begin{tikzpicture}[remember picture, overlay]
            \node[opacity = 0.2] at (current page.center)
            {
                \includegraphics[height = 0.75\paperwidth]{Images/bg3.png}
            };
        \end{tikzpicture}

        \normalsize
        \tableofcontents
    \end{frame}

    \section{Goal}

    \begin{frame}[t,allowframebreaks]{RSP framework}
    One of the most fundamental topics in network science is determining the \emph{centrality} of a node in a network accordingly to its structure. The concept of centrality can be interpreted in many ways and a vast number of measures have been proposed.\\
    \vspace{0.8em}
    We aim to introduce two new closely related betweenness centrality measures based on the Randomized Shortest Paths (RSP) framework:
    \begin{itemize}
        \item the simple RSP betweenness

        \item the RSP net betweenness
    \end{itemize}
    %\vspace{0.8em}
    % Then we show with real world examples the potential of these mesures in identifying interesting nodes of a network that more traditional methods might fail to notice.

    \newpage

    The RSP framework is based on \emph{Boltzmann probability distributions} over paths between the nodes of a network which focus on short, optimal paths, but give some probablity mass also to longer paths.\\
    \vspace{0.8em}
    The Boltzmann probabilities describe the probability that a thermodynamic system is in a particular state, given a certain \emph{energy} value of that system.\\
    \vspace{0.8em}
    Rather than the energy, we'd like to control the focus on optimal paths with an inverse temperature parameter $\beta$.
    \end{frame}

    \begin{frame}[t,allowframebreaks]{Usefulness}
    Measures based only on shortest paths or random walks alone often involve undesirable features:
    \begin{itemize}
        \item highly skewed betweenness score distribution in complex networks;

        \item not always realistic to consider that navigating agents would occur along only the shortest paths;

        \item random walks may depend heavely on local features of a graph, especially for large graphs.
    \end{itemize}

    \newpage

    Because of this, RSP's measures can help:
    \begin{itemize}
        \item in detecting bottlenecks of the networks, where there exist no alternatives for the shortest path;

        \item in introducing some regularisation over the degree of randomness, which is controlled by the inverse temperature parameter $\beta$.
    \end{itemize}
    \vspace{0.8em}
    In addiction, the RSP framework has previously been used for defining distance measures in many data analysis tasks such as clustering and classification of graph nodes.
    \end{frame}

    % \section{Notation}

    \section{RSP betweenness centralities}

    \begin{frame}[t,allowframebreaks]{Notation and terminology}

        \begin{itemize}
            \item  We consider weighted directed graphs $G=(V,E)$ with node set $V=\{1,2,\dots,n\}$ and edge set $E=\{(i,j)\}$ of $m$ edges.

            % \item We define a path as a sequence of nodes $\bondox{P}=(i_0, \dots, i_T)$.

            % \item We say that a path is \emph{absorbing} if the last node of the path appears on the path only once.

            % \item We denote the set of all absorbing \emph{s-t-path} by $\mathcal{P}_{st}$.

            \item Each edge of the graph is associated with a \textbf{weight} $a_{ij}$ and a \textbf{cost} $c_{ij}$:

            \begin{itemize}
                \item The weights are collected in the non-symmetric $n\times n$ \emph{adjacency matrix} $\mathbf{A}$ and they reflect the strength of connection between adjacent nodes.

                \item The edge weights define the \emph{reference transition probability matrix} $\mathbf{P}^{\text{ref}}$, which can be computed as
                \begin{equation*}
                \mathbf{P}^{\text{ref}}=\mathbf{D}^{-1}\mathbf{A},\qquad\mathbf{D}=\text{diag}(k_1,\dots,k_n)
                \end{equation*}
                or
                \begin{equation*}
                p^{\text{ref}}_{ij} = \frac{a_{ij}}{\sum_{j} a_{ij}} = \frac{a_{ij}}{s_i^{\text{out}}}
                \end{equation*}

                \item The edge costs, in contrast to weights, describe the distance of adjacent nodes. The cost of a path $\bondox{P}$ is
                \begin{equation*}
                \widetilde{c}(\bondox{P})=\sum_{(i,j)\in\bondox{P}} c_{ij}
                \end{equation*}
            \end{itemize}

            % \item We use the term \textbf{shortest path} to mean the path between two nodes with the lower cost over all paths between the nodes.

            % \item We denote the set of shortest paths from node $s$ to node $t$ by $\mathcal{P}^*_{st}$, the total number of such paths by $\left| \mathcal{P}^*_{st} \right|$ and the cost of the shortest path from $s$ to $t$ by $\widetilde{c}^*_{st}$.
        \end{itemize}

        In many situations the edge costs and edge weights can be defined based on one another, for istance, as $c_{ij}=1/a_{ij}$. 

        However, in the RSP framework the weights can be independent of the costs, thus the transition probabilities do not depend on the costs. This means that:

        \begin{itemize}
            \item[$\triangleright$] the edge costs define the interpretation of shortest paths, i.e. the \emph{low temperature behavior} of the system;

            \item[$\triangleright$] the edge weights determine the interpretation of a random walk, i.e. the \emph{high temperature behavior}.
        \end{itemize}

    \end{frame}

    % \begin{frame}[t,allowframebreaks]{Betweenness centrality}
    % The best-known centrality measure of all is the original \emph{shortest path betweenness centrality} of Freeman, which counts the fraction of shortest paths between a pair of nodes that an intermediate node lies on and sum these fractions over all node pairs:
    % \begin{equation*}
    % b_i=\sum_{s,t=1}^n \frac{n(i\in\mathcal{P}^*_{st})}{\left| \mathcal{P}^*_{st} \right|} 
    % \end{equation*}
    % where $n(i\in\mathcal{P}^*_{st})$ means the number of paths that contain node $i$.

    % % ......... tutte le altre

    % \end{frame}

    \begin{frame}[t,allowframebreaks]{Minimization of expected cost}

    For semplicity, we restrict the RSP framework to absorbing paths (for istance, take a directed strongly connected network).\\
    \vspace{0.8em}
    The RSP framework is based on the probability distribution over the set $\mathcal{P}_{st}$ of absorbing $s$-$t$-walks for which the expected cost of the walk is minimal when constrained with a fixed relative entropy w.r.t. the reference path probability distribution. Formally, we seek for the solution to the following problem:
    \begin{equation*}
    \min_{\widetilde{P}_{st}}\sum_{\bondox{P}\in \mathcal{P}_{st}}\widetilde{P}_{st}(\bondox{P})\,\widetilde{c}(\bondox{P})\quad\text{s.t.}\quad \begin{cases}
    J\left(\widetilde{P}_{st}||\widetilde{P}_{st}^{\text{ref}}\right)=J_0 \\
    \displaystyle \sum_{\bondox{P}\in \mathcal{P}_{st}} \widetilde{P}_{st}(\bondox{P}) = 1
    \end{cases}
    \end{equation*}

    % where $\widetilde{P}_{st}(\bondox{P})$ is the reference path probability, $\widetilde{c}(\bondox{P})$ the overall cost of path $\bondox{P}$ and $J\left(\widetilde{P}_{st}||\widetilde{P}_{st}^{\text{ref}}\right)$ is the relative entropy (Kullback-Leibler divergence).

    The solution is a Boltzmann distribution:
    \begin{equation*}
    \widetilde{P}_{st}(\bondox{P})=\frac{\widetilde{P}_{st}^{\text{ref}}(\bondox{P})\,e^{-\beta\widetilde{c}(\bondox{P})}}{\mathcal{Z}_{st}} 
    \end{equation*}
    where
    \begin{equation*}
    \mathcal{Z}_{st}=\sum_{\bondox{P}\in \mathcal{P}_{st}}\widetilde{P}_{st}^{\text{ref}}(\bondox{P})\,e^{-\beta\widetilde{c}(\bondox{P})}
    \end{equation*}
    is the \emph{partition function} of absorbing $s$-$t$-walks. %just a normalizing factor
    \end{frame}

    \begin{frame}[t,allowframebreaks]{Matrix formulation}
    Concerning the computation of $\mathcal{Z}_{st}$, we have that
    \begin{equation*}
    \mathcal{Z}_{st}=\frac{z_{st}}{z_{tt}} 
    \end{equation*}
    where $z_{st}$ is the $(s,t)$-element of the \emph{fundamental matrix of non-absorbing paths}
    \begin{equation*}
    \mathbf{Z}=(\mathbf{I}-\mathbf{W})^{-1}\quad\text{with}\quad \mathbf{W}=\mathbf{P}^{\text{ref}}\circ e^{-\beta\mathbf{C}}
    \end{equation*}
    %element-wise matrix multiplication
    The matrix $\mathbf{W}$ is a substochastic matrix and can be interpreted as defining a \emph{killed random walk}. Hence, one can see the partition function $\mathcal{Z}_{st}$ as the probability of a walker surviving the walk from $s$ to $t$.
    \end{frame}

    \begin{frame}[t,allowframebreaks]{Simple RSP betweenness}
    The simple RSP betweenness centrality of a node $i$ is
    \begin{equation*}
    \text{bet}_i^{\text{RSP}}=\sum_{s,t=1}^n \text{bet}_i^{\text{RSP}}(s,t)
    \end{equation*}
    where $\text{bet}_i^{\text{RSP}}(s,t)$ is the RSP simple betweenness of the node $i$ w.r.t. absorbing paths from $s$ to $t$, i.e. the expected number of visits through $i$ over all $s$-$t$-walks w.r.t. the RSP solution probabilities:
    \begin{equation*}
    \text{bet}_i^{\text{RSP}}(s,t)=
    \begin{cases}
    0 &\text{if }\nexists\ s\text{-}t\text{-path}\\
    \displaystyle\sum_{j=1}^n\overline{\eta}_{ij}(s,t) &\text{otherwise}
    \end{cases}
    \end{equation*} 
    where 
    \begin{equation*}
    \overline{\eta}_{ij}(s,t)=-\frac{1}{\beta}\,\frac{\partial\log \mathcal{Z}_{st}}{\partial c_{ij}}=-\frac{1}{\beta}\,\frac{\partial\log\left(z_{st}/z_{tt}\right) }{\partial c_{ij}}=\left(\frac{z_{si}}{z_{st}}-\frac{z_{ti}}{z_{tt}}\right)w_{ij}z_{jt}
    \end{equation*}
    In other words, the total flow transiting through node $i$ is
    \begin{equation*}
    \text{bet}_i^{\text{RSP}}(s,t)=\underbracket[0.5pt]{\left(\frac{z_{si}}{z_{st}}-\frac{z_{ti}}{z_{tt}}\right)}_{=0\text{ if }i=t}\underbracket[0.5pt]{\sum_{j} w_{ij}z_{jt}}_{\begin{gathered}
    \scriptstyle{=z_{it}\text{ if }i\neq t} \\
    \scriptstyle{\mathbf{Z}=(\mathbf{I}-\mathbf{W})^{-1}} \\
    \scriptstyle{\mathbf{Z}(\mathbf{I}-\mathbf{W})=\mathbf{I}} \\
    \scriptstyle{\mathbf{Z}=\mathbf{ZW}+\mathbf{I}} \\
    \end{gathered}}=\left(\frac{z_{si}}{z_{st}}-\frac{z_{ti}}{z_{tt}}\right)z_{it}
    \end{equation*}

    Finally, the simple RSP betweenness of node $i$ is
    \begin{align*}
    \text{bet}_i^{\text{RSP}}&=\sum_{s,t}\text{bet}_i^{\text{RSP}}(s,t)=\\
    &=\sum_{s,t} \left(\frac{z_{si}}{z_{st}}-\frac{z_{ti}}{z_{tt}}\right)z_{it}=\\
    &=\left[ \textbf{diag}\left( \mathbf{Z} \left(\mathbf{Z}^\div-n \textbf{Diag}\left( \mathbf{Z}^\div \right) \right)^T\mathbf{Z} \right) \right]_i
    \end{align*}

    %z_ij div = 1 / z_ij

    The vector $\textbf{bet}^\text{RSP}$ of all betweenness values is computed accordingly.
    \end{frame}

    \begin{frame}[t,allowframebreaks]{Computing the simple RSP betweenness}
    \textbf{Input:}
    \begin{itemize}
        \item a directed strongly connected graph $G$ with $n$ nodes
        \item the $n\times n$ reference transition probability matrix $\mathbf{P}^\text{ref}=\mathbf{D}^{-1}\mathbf{A}$
        \item the $n\times n$ cost matrix $\mathbf{C}$
        \item the inverse temperature parameter $\beta$
    \end{itemize}

    \textbf{Output:}
    \begin{equation*}
    \begin{array}{cl}
    1. & \mathbf{W}=\mathbf{P}^\text{ref}\circ e^{-\beta\mathbf{C}} \\
    2. & \mathbf{Z}=(\mathbf{I}-\mathbf{W})^{-1} \\
    3. & \mathbf{Z}^\div=\mathbf{e}\mathbf{e}^T\div\mathbf{Z} \\
    4. & \text{return }\textbf{bet}^\text{RSP}=\textbf{diag}\left( \mathbf{Z} \left(\mathbf{Z}^\div-n \textbf{Diag}\left( \mathbf{Z}^\div \right) \right)^T\mathbf{Z} \right)
    \end{array}
    \end{equation*}

    This highlight the computational bottleneck of the algorithm: the matrix inversion, which, in general, has complexity $\mathcal{O}\left( n^3 \right)$. % why?
    \end{frame}

    \begin{frame}[t,allowframebreaks]{RSP net betweenness}
    Instead of only considering the overall outgoing flow of random walkers it may in some cases make more sense to compute the net outgoing flow, i.e. so that the outgoing and ingoing flows through one edge neutralize each other. This corresponds to the random walk interpretation of the \emph{current flow betweenness} in undirected graphs.\\% why???????????????
    \vspace{0.8em} 
    According to this approach, we define the \emph{RSP net betweenness centrality} of node $i$ as
    \begin{equation*}
    \text{bet}_i^{\text{RSPnet}}=\sum_{s,t}
    \sum_{j}\left|\overline{\eta}_{ij}(s,t)-\overline{\eta}_{ji}(s,t)\right|
    \end{equation*}
    % la forma chiusa è più tecnica ma non più difficile, mentre l'algoritmo ...

    \end{frame}

    \section{Limits and Experiments}

    \begin{frame}[t,allowframebreaks]{Limits}
    \begin{equation*}
    \widetilde{P}_{st}(\bondox{P})=\frac{\widetilde{P}_{st}^{\text{ref}}(\bondox{P})\,e^{-\beta\widetilde{c}(\bondox{P})}}{\mathcal{Z}_{st}},\qquad 
    \mathcal{Z}_{st}=\sum_{\bondox{P}\in \mathcal{P}_{st}}\widetilde{P}_{st}^{\text{ref}}(\bondox{P})\,e^{-\beta\widetilde{c}(\bondox{P})}
    \end{equation*}
    
    \begin{itemize}
        \item At the limit $\beta\to\infty$. In the low temperature limit, the RSP probability distribution focuses solely on shortest paths. Thus 
        \begin{equation*}
        \widetilde{P}_{st}(\bondox{P})\xrightarrow{\beta\to\infty}\begin{cases}
            0, &\ \text{ if }\bondox{P}\notin \mathcal{P}^*_{st}\\
            \dfrac{\widetilde{P}_{st}^{\text{ref}}(\bondox{P})}{\displaystyle \sum_{\bondox{P}\in\mathcal{P}^*_{st}} \widetilde{P}_{st}^{\text{ref}}(\bondox{P})} &\ \text{ if }\bondox{P}\in \mathcal{P}^*_{st}
        \end{cases}
        \end{equation*}

        % spiegare perché

        In other words, the simple RSP  betweenness converges to the \emph{shortest path likelihood betweenness}.

        The same result holds for the RSP net betweenness. Intuitively, as the path distribution focuses more and more on the shortest paths, one of the two terms of in and outgoing flows becomes zero, as the walker will only move in one direction.

        \item At the limit $\beta\to0^+$. In the high temperature limit the RSP probabilities converge to the \emph{unbiased random walk probabilities}, determined by the reference transition probabilities, i.e. % cosa vuol dire beta to zero (intuitivamente è un random walk se non si guardano più gli shortest path)
        \begin{equation*}
        \widetilde{P}_{st} \xrightarrow{\beta\to0^+}\widetilde{P}_{st}^{\text{ref}}\qquad\forall\, \bondox{P}
        \end{equation*}
        This means that the simple RSP betweenness is proportional to the stationary distribution $\pi$ s.t. $\pi=\left( \mathbf{P}^\text{ref} \right)^T\pi$.

        %  converges to the expected number of visits to a node over all absorbing walks with respect to the unbiased random walk probabilities; this measure, if the network is strongly connected and aperiodic, is proportional to the stationary distribution $\pi$ s.t. $\pi=\left( \mathbf{P}^\text{ref} \right)^T\pi$, thus for undirected networks the measure becomes proportional to the degree centrality. % or strenght centrality

        % % whyyyyyy?????

        % In the same limit the RSP net flow converges to the \emph{current flow betweenness centrality}. %cos'è???
    \end{itemize}

    \end{frame}

    \begin{frame}[t,allowframebreaks]{Artificial example}
    One possible use for the RSP betweenness measures is the detection of groups of nodes that are central in a network.

    % esempio delle comunità A-B-C in un social network

    % The graph has been constructed by considering three cliques of 6 nodes, and by removing and adding appropriate links to obtain inter-community communication.

    \begin{figure}
    \includegraphics[height =0.6\paperheight]{Images/heat1.png}
    \end{figure}

    % (a) is likelihood EQUIV betweenness che abbiamo studiato noi
    % (c) is the stationary distribution multiplied by the sum of average hitting times, which – as the network is undirected – corresponds to the degree centrality up to a scaling factor.
    % The heat plots show that the simple RSP betweenness highlights the nodes of the central community more than its limit functions. The low temperature limit function, i.e. the shortest path likelihood betweenness high- lights the nodes connecting the different communities, but the betweenness scores of the two other nodes in the central community are of the same magnitude as the scores of the other nodes in the peripheral communities. 

    % no benefit nell'usare una cosa complicata come la RSP net betweenness
    \end{frame}

    \begin{frame}[t,allowframebreaks]{Manhattan street network.}
    One promising application area for RSP’s are path planning problems. % why? RSP’s allow the modeling of routing in situations that include an element of randomness, such as navigation of people. RSP’s can be used for planning paths in an optimal way while keeping the predictability of the path at a desired level.

    \vspace{0.8em}

    We illustrate the use of RSP’s for routing in a network by analyzing the street network of Midtown and Lower Manhattan.

    \begin{center}
    \begin{figure}
    \includegraphics[height=0.4\paperheight]{Images/man.jpg}
    \end{figure}
    \end{center}

    \newpage

    The nodes in the network correspond to intersections and the edges are the street segments between the intersections; we treat the network as undirected. 

    \vspace{0.8em}

    The length of each street segment is assigned as the cost of the corresponding edge. Accordingly, the overall
    cost of a path is its overall length. However, we define here the reference transition probabilities of the random
    walk according to the degree of each node, $p_{ij}^\text{ref}=1/k_i$, i.e. only according to the number of edges connected to the node and independent of the edge costs.

    % This seems like a reasonable choice for moving in a street network, if we consider that the decision of direction of the random walker in an intersection is not affected by the lengths of the street segments. Remember that this means that the shortest path likelihood betweenness on the graph is based on the edge costs, whereas the random walk based betweenness measures do not depend on the costs, but only the degrees of nodes.

    \newpage

    \begin{center}
    \begin{figure}
    \includegraphics[height=0.65\paperheight]{Images/heat2.png}
    \end{figure}
    \end{center}

    % At low temperatures (large $\beta$), both RSP based betweenness measures converge to the shortest path likelihood betweenness. As the network is undirected and connected, and because we use reference probabilities based only on degrees, instead of costs, the limit function of the simple RSP betweenness, when $\beta\to0^+$, is equal to the degree centrality multiplied by a constant.

    % It is evident from the plot that the shortest path betweenness is focused on Broadway, which functions as a diagonal shortcut in many routes on the grid-like Midtown. However, when β is decreased, both RSP based measures rank highest the intersections along the FDR Drive on the eastern shore. This is mainly caused by the sparsity of streets on the east shore close to the residential areas of Stuyvesant Town and Peter Cooper Village. As a result, the FDR Drive is a vital connection between the upper and lower eastern parts of the map. This aspect is not clear from the shortest path likelihood betweenness, but becomes apparent by computing the RSP-based betweenness values. The current flow between- ness also ranks high the intersections of FDR Drive, but as a drawback the importance of Broadway is not as apparent as it perhaps deserves.

    % Thus, it seems that the RSP based betweenness measures, by assuming suboptimal navigation between points in the network, can highlight bottlenecks such as the FDR Drive on the Manhattan network better than the deterministic shortest path betweenness or the unbiased current flow betweenness.
    \end{frame}









    % \section{Old}

    % \setbeamercolor{section in head/foot}{bg=cyan,fg=black}
    % \setbeamercolor{frametitle}{bg=aqua,fg=black}
    % \setbeamercolor{author in head/foot}{bg=aqua,fg=black}

    % \begin{frame}[t,allowframebreaks]{Types of Graph}
    %     \begin{columns}
    %         \column{0.5\textwidth}
    %         \begin{bee}[Undirected Graph,width = 6cm]{black}{aqua}
    %             A graph in which edges do not have any direction.
    %         \end{bee}
    %         \column{0.55\textwidth}
    %         \begin{Rbee}[Directed Graph,width = 6cm]{arsenic}{aqua}
    %             A graph in which edge has direction.
    %         \end{Rbee}
    %     \end{columns}
    %     \begin{figure}
    %             \includegraphics[height =0.4\paperheight]{Images/diagram-20240216.png}
    %     \end{figure}


    %     \begin{bee}[Complete Graph,width = 8cm]{white}{deepmagenta}
    %         A graph in which every pair of distinct nodes is connected by an edge
    %     \end{bee}
    %        \vspace{-5mm}
    %     \begin{Rbee}[Forest,width = 0.86\paperwidth]{white}{arsenic}
    %         A collection of trees or disjoint tree-like structures within a graph
    %     \end{Rbee}
    %     \begin{bee}[Tree,width = 0.9\paperwidth]{white}{blue}
    %         A special case of an acyclic graph in which there is a single root node, and every other node is connected by exactly one edge.
    %     \end{bee}

    % \end{frame}

    % \begin{frame}[t,allowframebreaks]{Introduction to Graphs}

    % \begin{itemize}
    %     \item  A Graph is a non-linear data structure consisting of vertices and edges.
    %     \item The \textbf{Vertices} are sometimes also referred to as nodes and the \textbf{Edges} are lines or arcs that connect any two nodes in the graph.

    %     \begin{bee}[More formally]{white}{arsenic}
    %         A Graph is composed of a set of vertices V and a set of edges E. \\
    %         The graph is denoted by \textbf{G(V,E)}.
    %     \end{bee}

    % \end{itemize}

    % \end{frame}

\end{document}
